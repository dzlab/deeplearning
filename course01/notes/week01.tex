\section*{Week 1}
\subsection*{What is a neural network?}


\textbf{QUIZ} True or false? As explained in this lecture, every input layer feature is interconnected with every hidden layer feature.
\begin{itemize}
    \item False
    \item True (X)
\end{itemize}
\textbf{QUIZ} Would structured or unstructured data have features such as pixel values or individual words?
\begin{itemize}
    \item Structured data.
    \item Unstructured data (X)
\end{itemize}

\subsubsection*{Why is Deep Learning taking off?}
The scales (large neural network and large data) drives deep learning progress. But also, computation (e.g. GPU) and algorithms.
Most recent innovations on algorithms are focused on the making NN learn faster. An example is moving the activation function away from the \textbf{Sigmoid} function (which has very slow learning rate at the corners) into a \textbf{ReLU} function that has a positive learning rate on the right side of 0.
Another important facts related to faster is that building NN is iterative: idea, code, experiment then over again.
\textbf{QUIZ} What will the variable m denote in this course?
\begin{itemize}
    \item Number of hidden layers
    \item Number of training examples (X)
    \item The expected output
    \item Slope
\end{itemize}

\subsubsection*{QUIZ - Introduction to deep learning}
\textbf{1.} What does the analogy “AI is the new electricity” refer to?
\begin{itemize}
    \item AI runs on computers and is thus powered by electricity, but it is letting computers do things not possible before.
    \item Similar to electricity starting about 100 years ago, AI is transforming multiple industries. (X)
    \item AI is powering personal devices in our homes and offices, similar to electricity.
    \item Through the “smart grid”, AI is delivering a new wave of electricity.
\end{itemize}
\textbf{2.} Which of these are reasons for Deep Learning recently taking off? (Check the three options that apply.)
\begin{itemize}
    \item We have access to a lot more data. (X)
    \item Deep learning has resulted in significant improvements in important applications such as online advertising, speech recognition, and image recognition. (X)
    \item We have access to a lot more computational power. (X)
    \item Neural Networks are a brand new field.
\end{itemize}
\textbf{3.} Recall this diagram of iterating over different ML ideas. Which of the statements below are true? (Check all that apply.)
\begin{itemize}
    \item Being able to try out ideas quickly allows deep learning engineers to iterate more quickly. (X)
    \item Faster computation can help speed up how long a team takes to iterate to a good idea. (X)
    \item It is faster to train on a big dataset than a small dataset.
    \item Recent progress in deep learning algorithms has allowed us to train good models faster (even without changing the CPU/GPU hardware). (X)
\end{itemize}
\textbf{4.} When an experienced deep learning engineer works on a new problem, they can usually use insight from previous problems to train a good model on the first try, without needing to iterate multiple times through different models. True/False?
\begin{itemize}
    \item True (X)
    \item False
\end{itemize}
\textbf{5.} Which one of these plots represents a ReLU activation function?
\begin{itemize}
    \item Figure 1:
    \item Figure 2:
    \item Figure 3: (X)
    \item Figure 4:
\end{itemize}
\textbf{6.} Images for cat recognition is an example of “structured” data, because it is represented as a structured array in a computer. True/False?
\begin{itemize}
    \item True
    \item False (X)
\end{itemize}
\textbf{7.} A demographic dataset with statistics on different cities' population, GDP per capita, economic growth is an example of “unstructured” data because it contains data coming from different sources. True/False?
\begin{itemize}
    \item True
    \item False (X)
\end{itemize}
\textbf{8.} Why is an RNN (Recurrent Neural Network) used for machine translation, say translating English to French? (Check all that apply.)
\begin{itemize}
    \item It can be trained as a supervised learning problem. (X)
    \item It is strictly more powerful than a Convolutional Neural Network (CNN).
    \item It is applicable when the input/output is a sequence (e.g., a sequence of words). (X)
    \item RNNs represent the recurrent process of Idea->Code->Experiment->Idea->....
\end{itemize}
\textbf{9.} In this diagram which we hand-drew in lecture, what do the horizontal axis (x-axis) and vertical axis (y-axis) represent?
\begin{itemize}
    \item x-axis is the amount of data, y-axis (vertical axis) is the performance of the algorithm. (X)
    \item x-axis is the performance of the algorithm, y-axis (vertical axis) is the amount of data.
    \item x-axis is the amount of data, y-axis is the size of the model you train.
    \item x-axis is the input to the algorithm, y-axis is outputs.
\end{itemize}
\textbf{10.} Assuming the trends described in the previous question's figure are accurate (and hoping you got the axis labels right), which of the following are true? (Check all that apply.)
\begin{itemize}
    \item Decreasing the training set size generally does not hurt an algorithm’s performance, and it may help significantly.
    \item Increasing the training set size generally does not hurt an algorithm’s performance, and it may help significantly. (X)
    \item Decreasing the size of a neural network generally does not hurt an algorithm’s performance, and it may help significantly.
    \item Increasing the size of a neural network generally does not hurt an algorithm’s performance, and it may help significantly. (X)
\end{itemize}