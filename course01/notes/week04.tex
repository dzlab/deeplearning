\section*{Week 4}
\subsection*{Deep Neural Network}

\subsubsection*{Getting your matrix dimensions right}
\begin{tikzpicture}[shorten >=1pt,->,draw=black!50, node distance=\layersep]
    \tikzstyle{every pin edge}=[<-,shorten <=1pt]
    \tikzstyle{neuron}=[circle,fill=black!25,minimum size=17pt,inner sep=0pt]
    \tikzstyle{input neuron}=[neuron, fill=green!50];
    \tikzstyle{output neuron}=[neuron, fill=red!50];
    \tikzstyle{hidden neuron_1}=[neuron, fill=blue!50];
    \tikzstyle{hidden neuron_2}=[neuron, fill=blue!50];
    \tikzstyle{hidden neuron_3}=[neuron, fill=blue!50];
    \tikzstyle{hidden neuron_4}=[neuron, fill=blue!50];
    \tikzstyle{annot} = [text width=4em, text centered]

    % Draw the input layer nodes
    \foreach \name / \y in {1,...,2}
    % This is the same as writing \foreach \name / \y in {1/1,2/2,3/3,4/4}
        \node[input neuron, pin=left:Input \#\y] (I-\name) at (0,-\y) {};

    % Draw the hidden layer (1) nodes
    \foreach \name / \y in {1,...,3}
        \path[yshift=0.5cm]
            node[hidden neuron_1] (H1-\name) at (\layersep,-\y cm) {};

    % Draw the hidden layer nodes
    \foreach \name / \y in {1,...,5}
        \path[yshift=0.5cm]
            node[hidden neuron_2] (H2-\name) at (2*\layersep,-\y cm) {};

    % Draw the hidden layer nodes
    \foreach \name / \y in {1,...,4}
        \path[yshift=0.5cm]
            node[hidden neuron_3] (H3-\name) at (3*\layersep,-\y cm) {};

    % Draw the hidden layer nodes
    \foreach \name / \y in {1,...,2}
        \path[yshift=0.5cm]
            node[hidden neuron_4] (H4-\name) at (4*\layersep,-\y cm) {};

    % Draw the output layer node
    \node[output neuron,pin={[pin edge={->}]right:Output}, right of=H4-2] (O) {};

    % Connect every node in the input layer with every node in the
    % hidden layer.
    \foreach \source in {1,...,2}
        \foreach \dest in {1,...,3}
            \path (I-\source) edge (H1-\dest);

    % Connect every node in the input layer with every node in the
    % hidden layer.
    \foreach \source in {1,...,3}
        \foreach \dest in {1,...,5}
            \path (H1-\source) edge (H2-\dest);

    % Connect every node in the input layer with every node in the
    % hidden layer.
    \foreach \source in {1,...,5}
        \foreach \dest in {1,...,4}
            \path (H2-\source) edge (H3-\dest);

    % Connect every node in the input layer with every node in the
    % hidden layer.
    \foreach \source in {1,...,4}
        \foreach \dest in {1,...,2}
            \path (H3-\source) edge (H4-\dest);

    % Connect every node in the hidden layer with the output layer
    \foreach \source in {1,...,2}
        \path (H4-\source) edge (O);

    % Annotate the layers
    \node[annot,above of=H1-1, node distance=1cm] (hl) {Hidden layer 1};
    \node[annot,above of=H2-1, node distance=1cm] (h2) {Hidden layer 2};
    \node[annot,above of=H3-1, node distance=1cm] (h3) {Hidden layer 3};
    \node[annot,above of=H4-1, node distance=1cm] (h4) {Hidden layer 4};
    \node[annot,left of=hl] {Input layer};
    \node[annot,right of=h4] {Output layer};
\end{tikzpicture}
% End of code

With $l$ denoting the layer number, $n^{[l]}$ the number of nodes at layer $l$:

\begin{equation}
    Z^{[l]} = W^{[l]} * A^{[l-1]} + b^{[l]}
\end{equation}
The dimensions of the matrices $W$ and $b$ for one sample are:
\begin{equation}
    W^{[l]} : (n^{[l]}, n^{[l-1]})
\end{equation}
\begin{equation}
    b^{[l]} : (n^{[l]}, 1)
\end{equation}
For $m$ samples:
\begin{equation}
    Z^{[l]} : (n^{[l]}, m)
\end{equation}
\begin{equation}
    A^{[l-1]} : (n^{[l-1]}, m)
\end{equation}
\begin{equation}
    b^{[l]} : (n^{[l]}, 1)
\end{equation}

\subsubsection*{Backward propagation for layer l}

Input $da^{[l]}$
Output $da^{[l-1]}, dW^{[l]}, db^{[l]}$
Then
\begin{equation}
dz^{[l]} = da^{[l]} * g^{[l]'}(z^{[l]})
\end{equation}
\begin{equation}
dw^{[l]} = dz^{[l]} * a^{[l-1]}
\end{equation}
\begin{equation}
db^{[l]} = dz^{[l]}
\end{equation}
\begin{equation}
da^{[l-1]} = W^{[l]T} dz^{[l]}
\end{equation}
Then $dz^{[l]}$ becomes
\begin{equation}
dz^{[l]} = w^{[l+1]T} dz^{[l+1]} * g^{[l]'}(z^{[l]})
\end{equation}
The vectorized version will looks like:
\begin{equation}
dZ^{[l]} = dA^{[l]} * g^{[l]'}(Z^{[l]})
\end{equation}
\begin{equation}
dW^{[l]} = \frac{1}{m} dZ^{[l]} . A^{[l-1]T}
\end{equation}
\begin{equation}
db^{[l]} = \frac{1}{m} np.sum(dZ^{[l]}, axis = 1, keepdims = True)
\end{equation}
\begin{equation}
dA^{[l-1]} = W^{[l]T} . dZ^{[l]}
\end{equation}


\subsubsection*{What are hyperparameters?}

Parameters: $W^{[1]}, b^{[1]}, W^{[2]}, b^{[2]}, W^{[3]}, b^{[3]} ...$
Hyperparameters: paramters that determines the raw parameters above
\begin{itemize}
    \item learning rate $\alpha$
    \item number of iterations
    \item number of hidden layers L
    \item number of hidden units $n^{[1]}, n^{[2]}, ...$
    \item choice of activation layers
\end{itemize}

Later more hyperprameters: Momentum, mini-batch size, regulations


\subsubsection*{QUIZ - Key concepts on Deep Neural Networks}
\textbf{1.} What is the "cache" used for in our implementation of forward propagation and backward propagation?
\begin{itemize}
    \item We use it to pass variables computed during forward propagation to the corresponding backward propagation step. It contains useful values for backward propagation to compute derivatives. (X)
    \item We use it to pass variables computed during backward propagation to the corresponding forward propagation step. It contains useful values for forward propagation to compute activations.
    \item It is used to cache the intermediate values of the cost function during training.
    \item It is used to keep track of the hyperparameters that we are searching over, to speed up computation.
\end{itemize}
\textbf{2.} Among the following, which ones are "hyperparameters"? (Check all that apply.)
\begin{itemize}
    \item number of iterations (X)
    \item activation values $a^{[l]}$
    \item number of layers LL in the neural network (X)
    \item weight matrices $W^{[l]}$
    \item size of the hidden layers $n^{[l]}$ (X)
    \item bias vectors $b^{[l]}$
    \item learning rate $\alpha$ (X)
\end{itemize}
\textbf{3.} Which of the following statements is true?
\begin{itemize}
    \item The deeper layers of a neural network are typically computing more complex features of the input than the earlier layers. (X)
    \item The earlier layers of a neural network are typically computing more complex features of the input than the deeper layers.
\end{itemize}
\textbf{4.} Vectorization allows you to compute forward propagation in an LL-layer neural network without an explicit for-loop (or any other explicit iterative loop) over the layers l=1, 2, …,L. True/False?
\begin{itemize}
    \item True
    \item False (X)
\end{itemize}
\textbf{5.} Assume we store the values for $n^{[l]}$ in an array called layers, as follows: $layer_dims = [n_x, 4,3,2,1]$. So layer 1 has four hidden units, layer 2 has 3 hidden units and so on. Which of the following for-loops will allow you to initialize the parameters for the model?
\begin{lstlisting}
for(i in range(1, len(layer_dims))):
  parameter['W' + str(i)] = np.random.randn(layers[i], layers[i-1])) * 0.01
  parameter['b' + str(i)] = np.random.randn(layers[i], 1) * 0.01
\end{lstlisting}
\textbf{6.} Consider the following neural network.

How many layers does this network have?
\begin{itemize}
    \item The number of layers LL is 4. The number of hidden layers is 3. (X)
    \item The number of layers LL is 3. The number of hidden layers is 3.
    \item The number of layers LL is 4. The number of hidden layers is 4.
    \item The number of layers LL is 5. The number of hidden layers is 4.
\end{itemize}
\textbf{7.} During forward propagation, in the forward function for a layer ll you need to know what is the activation function in a layer (Sigmoid, tanh, ReLU, etc.). During backpropagation, the corresponding backward function also needs to know what is the activation function for layer ll, since the gradient depends on it. True/False?
\begin{itemize}
    \item True
    \item False (X)
\end{itemize}
\textbf{8.} There are certain functions with the following properties:

(i) To compute the function using a shallow network circuit, you will need a large network (where we measure size by the number of logic gates in the network), but (ii) To compute it using a deep network circuit, you need only an exponentially smaller network. True/False?
\begin{itemize}
    \item True (X)
    \item False
\end{itemize}
\textbf{9.} Consider the following 2 hidden layer neural network:
Which of the following statements are True? (Check all that apply).
\begin{itemize}
    \item $W^{[1]}$ will have shape (4, 4) (X)
    \item $b^{[1]}$ will have shape (4, 1) (X)
    \item $W^{[1]}$ will have shape (3, 4)
    \item $b^{[1]}$ will have shape (3, 1)
    \item $W^{[2]}$ will have shape (3, 4) (X)
    \item $b^{[2]}$ will have shape (1, 1)
    \item $W^{[2]}$ will have shape (3, 1)
    \item $b^{[2]}$ will have shape (3, 1) (X)
    \item $W^{[3]}$ will have shape (3, 1)
    \item $b^{[3]}$ will have shape (1, 1) (X)
    \item $W^{[3]}$ will have shape (1, 3) (X)
    \item $b^{[3]}$ will have shape (3, 1)
\end{itemize}
\textbf{10.} Whereas the previous question used a specific network, in the general case what is the dimension of $W^{[l]}$, the weight matrix associated with layer ll?
\begin{itemize}
    \item $W^{[l]}$ has shape $(n^{[l]}, n^{[l-1]})$ (X)
    \item $W^{[l]}$ has shape $(n^{[l]}, n^{[l+1]})$
    \item $W^{[l]}$ has shape $(n^{[l-1]}, n^{[l]})$
    \item $W^{[l]}$ has shape $(n^{[l+1]}, n^{[l]})$
\end{itemize}